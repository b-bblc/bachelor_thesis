\documentclass{article}
\usepackage{titlesec}
\usepackage[german,english]{babel}
\usepackage[utf8]{inputenc}
\usepackage{fancyhdr}
\usepackage{graphicx}
\usepackage[round]{natbib}
\usepackage[onehalfspacing]{setspace}
\usepackage{subfiles}
\usepackage{float}
\usepackage[
    colorlinks=true,
    linkcolor=black,
    urlcolor=blue,
    citecolor=black
]{hyperref}
\usepackage{todonotes}
\usepackage[hang]{footmisc}
\setlength\footnotemargin{10pt}
\usepackage{amsmath}
\usepackage{listings}

% citation abbreviations
\newcommand{\citepage}[2]{ (\citet{#1}, \textit{p.}\,#2)}
\newcommand{\citepages}[2]{ (\citet{#1}, \textit{pp.}\,#2)}
\newcommand{\page}[1]{ (\textit{p.}\,#1)}
\newcommand{\q}[1]{``#1''}
\newcommand{\dashitem}{\item[\textendash]}

% page layout
\pagestyle{fancy}
\fancyhf{}
\fancyhead[L]{\leftmark}
\fancyfoot[C]{\thepage}

\title{
    \large{
        University of Potsdam\\Department of Linguistics\\[1cm]
        \textbf{\Large{Dependency Structures in Elementary Discourse Units}}\\[1cm]
        Bachelor Thesis in partial fulfillment of the requirements for the degree of\\
        \textbf{Bachelor of Science in Computational Linguistics}\\[1cm]
        \textbf{Artur Begichev}\\
        begichev1@uni-potsdam.de\\
        Matriculation number: 805310\\[1cm]
        Supervisors:\\[0.3cm]
        Prof. Manfred Stede\\University of Potsdam\\[0.3cm]
        Prof. Maciej Ogrodniczuk\\Institute of Computer Science, Polish Academy of Sciences\\
    }
}

\begin{document}
\pagestyle{empty}
\maketitle
\thispagestyle{empty}
\newpage

\begin{otherlanguage}{german}
\begin{abstract}
Diese Studie untersucht mithilfe automatischer Abhängigkeitsanalyse und Clusterverfahren die Organisation von Elementary Discourse Units (EDUs) in deutschen und russischen Korpora und identifiziert vier systematische Strukturmuster mit ausgeprägter Sprachspezifik (85,7\%). Die Ergebnisse zeigen, dass typologische Unterschiede (deutsche koordinations- und präpositionsreiche vs. russische nominaldichte und verbzentrierte Strukturen) die Diskursorganisation maßgeblich prägen und liefern praxisrelevante Impulse für computergestützte Diskursanalyse, maschinelle Übersetzung und Sprachdidaktik.
\end{abstract}
\end{otherlanguage}

\newpage
\tableofcontents
\newpage
\pagestyle{fancy}
\setcounter{page}{1}

\subfile{introduction}
\subfile{literature_review}
\subfile{materials_and_methods}
\subfile{implementation}
\subfile{results}
\subfile{discussion}
\subfile{conclusion}

\subfile{appendix}

% ===== References =====
\addcontentsline{toc}{section}{References}
\nocite{*} %
\bibliographystyle{apalike}
\bibliography{references}

\newpage
\thispagestyle{empty}
\subfile{declaration.tex}

\end{document}

\section{Conclusion}
This bachelor thesis presents a comprehensive computational investigation of dependency structures in Elementary Discourse Units (EDUs) across German and Russian, addressing fundamental questions about discourse organization patterns and cross-linguistic variation. Through analysis this study demonstrates that computational methods successfully reveal systematic patterns in discourse organization while uncovering significant language-specific preferences that challenge universal theories of discourse structure.

The research employed a methodological framework combining dependency parsing with spaCy models, comprehensive feature extraction, and unsupervised clustering to investigate two primary research questions: whether computational methods can discover interpretable patterns in EDU organization, and the extent to which such patterns are language-specific versus universal. The results provide definitive answers to both questions while establishing methodological foundations for computational cross-linguistic discourse analysis.

\subsection{Successful Computational Pattern Discovery}

The analysis demonstrates that automatic dependency parsing reveals meaningful patterns in Elementary Discourse Unit organization. Four distinct and interpretable clusters were discovered, corresponding to recognizable discourse archetypes: nominal elaboration patterns (dense descriptive content), structural complexity patterns (deep syntactic embedding), fragmentary/segmented patterns (list-like structures with minimal predicates), and predicate-adverbial focus patterns (action-oriented units with circumstantial detail). The linguistic interpretability of these clusters validates that computational approaches capture genuine discourse organizational principles rather than statistical artifacts. The silhouette score of 0.105, while modest, is typical for linguistic clustering with gradient boundaries.

\subsection{Strong Language Specificity in Discourse Organization}

The analysis reveals exceptionally strong language specificity in EDU organizational patterns, with average language specificity of 85.7\% across clusters. No clusters met universality criteria (balanced representation below 70\% for either language), indicating fundamentally different preferences for organizing syntactic information within discourse units.

Statistical analysis (methodology detailed in Section~3.5.2) confirmed this through highly significant results with moderate to strong effect size. This specificity significantly exceeds expectations from corpus size or genre effects, indicating genuine typological influences aligned with known differences between Germanic and Slavic languages.

The absence of universal patterns challenges theoretical frameworks proposing language-independent discourse principles and supports approaches emphasizing typological-discourse interaction. Cross-linguistic analysis revealed significant differences in 13 of 15 continuous features, with particularly large effect sizes for preposition ratios and punctuation patterns.

\subsection{Methodological Innovation and Theoretical Implications}

The thesis establishes a novel and robust methodological framework for computational cross-linguistic discourse analysis. The combination of dependency parsing, comprehensive 23-feature extraction (covering structural, POS distribution, dependency relation, and complexity dimensions), and unsupervised clustering provides an objective and replicable approach to investigating discourse organization across typologically diverse languages.

Feature sensitivity analysis revealed that part-of-speech distributions carry the most discriminative information for identifying discourse organizational patterns, while dependency relation features provide minimal additional discriminative power. This finding has important implications for understanding which aspects of syntactic structure are most relevant for discourse organization, suggesting that grammatical category balance within EDUs captures fundamental organizational principles.

The methodology's success in revealing linguistically meaningful patterns across Germanic German and Slavic Russian demonstrates broader applicability to other language pairs and discourse phenomena. Comprehensive validation procedures, including bootstrap resampling (89\% average agreement in cluster assignments), cross-validation, and feature sensitivity analysis, confirm that the discovered patterns represent stable characteristics rather than methodological artifacts.


\subsection{Theoretical Contributions}
This research advances computational discourse analysis and cross-linguistic theory in three key areas. First, the documented 85.7\% language specificity challenges universalist discourse theories by providing empirical support for approaches emphasizing typological-discourse interaction over universal cognitive constraints. Second, the discovery of linguistically interpretable patterns through unsupervised methods validates using syntactic features as proxies for discourse structure, confirming the viability of computational discourse analysis. Third, the findings bridge typological and discourse research by demonstrating that computational methods can capture typologically consistent distinctions---German EDUs favor coordination-rich structures while Russian EDUs exhibit diverse patterns spanning nominal-dense and verb-centered organizations.

\subsection{Applied Implications}

The findings have immediate implications for natural language processing, particularly for developing language-specific discourse analysis systems and improving cross-linguistic transfer in machine translation. For educational applications, the documented patterns illuminate discourse transfer effects in second language acquisition, with relevance for academic writing instruction accounting for typological differences in organizational preferences.

Several limitations constrain generalizability: corpus size imbalance between languages, restriction to specific genres (news commentary and academic prose), and reliance on syntactic features while excluding semantic and pragmatic dimensions. Future research should address these through balanced corpora, genre diversification, integration of semantic and pragmatic features, expansion to additional language pairs, and advanced clustering methods (hierarchical, density-based, spectral) to reveal finer-grained patterns.

This study demonstrates that computational dependency parsing reveals systematic patterns in Elementary Discourse Unit organization while documenting exceptionally strong language specificity (85.7\%) that challenges universal discourse structure theories. The identification of four linguistically meaningful clusters through unsupervised methods validates dependency-based features as effective representations of discourse properties, showing that typological differences between Germanic and Slavic languages extend beyond sentence-level syntax to discourse organization. These findings support functional-typological approaches emphasizing language-structure interaction and provide a methodological framework for systematic cross-linguistic discourse investigation, advancing both theoretical understanding and contributing to more effective NLP systems and empirically grounded models of human discourse organization.
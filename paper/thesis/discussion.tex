\section{Discussion}

\subsection{Impact of Genre Differences}

One important aspect that must be taken into account when interpreting the results of this study is the genre difference between the two corpora under investigation. The Russian corpus consists predominantly of academic texts, while the German corpus is composed of reader comments from an online news platform. These two genres vary significantly in their communicative goals, structural conventions, and syntactic complexity.

Academic texts typically exhibit a high degree of syntactic cohesion, formal tone, and standardised sentence structures. In contrast, reader comments are often more informal, elliptical, and fragmented, with frequent deviations from normative syntax. These differences can influence the structure of Elementary Discourse Units (EDUs), particularly in terms of clause boundaries, dependency completeness, and the frequency of non-canonical structures.

This discrepancy may lead to genre-induced variation in the resulting dependency parses, which in turn affects the clustering patterns and statistical regularities discovered during the analysis. Therefore, any observed cross-linguistic differences must be interpreted with caution, as they may be confounded by genre-specific properties rather than language-internal syntactic features alone.

\subsection{Topic Variability Between Corpora}

Another factor that potentially impacts the comparability of results is the thematic divergence between the two corpora. The Russian academic texts are topically focused on linguistic theory and language analysis, while the German comments span a broader and more heterogeneous set of topics, including politics, economics, and personal opinions.

Topic variability can influence discourse structuring strategies. For instance, argumentative or narrative texts may display different coherence relations and EDU segmentations compared to explanatory or descriptive texts. This may further interact with genre effects, compounding the complexity of interpreting cross-linguistic trends.

\subsection{Future Work and Genre Control}

To mitigate these influences in future studies, it would be beneficial to control for genre by selecting corpora of similar communicative style and topical scope. Alternatively, genre-specific patterns could be modeled explicitly, e.g., by annotating each EDU with genre labels and analysing cluster behaviour per genre. This would allow for a more nuanced understanding of how discourse structures manifest across languages, independent of external genre-driven variation.

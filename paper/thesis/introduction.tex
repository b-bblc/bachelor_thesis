\section{Introduction}

The way humans organise their speech and writing has been a focus of study in both traditional linguistics and computational linguistics for a long time. When people talk or write, they do not simply string sentences together one after another. Instead, there is a specific structure in which different parts work together to create a single whole. At first glance, this may seem obvious; however, upon closer examination, it reveals considerable complexity. 

The goal of this thesis is to investigate how syntactic dependency information can improve our understanding of discourse organisation across languages. Determining whether patterns of EDU segmentation are universal or language-specific is essential for building multilingual NLP tools and for testing theoretical predictions about discourse structure.

This study adopts \emph{Rhetorical Structure Theory (RST)} (Mann \& Thompson, 1988), which segments discourse into \emph{Elementary Discourse Units (EDUs)} -- minimal text segments that participate in discourse relations. While RST has been widely applied to text analysis, the syntactic properties of EDUs across languages remain underexplored, particularly regarding how EDU boundaries interact with dependency structures in typologically distinct languages such as German and Russian.

The present research investigates the relationship between syntactic dependency structures and EDU organisational patterns across languages. Traditional discourse analysis relies on manual annotation and predefined theoretical categories, which prevents researchers from detecting novel patterns that exceed current frameworks. This raises the question of whether computational approaches can identify organised patterns in EDU structures that enhance or extend existing theories. While discourse structures have been investigated in detail within single languages, cross-linguistic studies remain lean, hindering our understanding of which discourse principles are universal and which are language-specific. Moreover, current methods typically analyse syntactic and discourse structures separately, leaving the potential of dependency parsing to support discourse-level analysis largely underexplored.

This study examines the dependency syntax of EDU in German and Russian. At its core, the research asks how similar or different the syntactic dependency structures of EDUs are in these two languages when analysed through parsing methods. A further goal is to examine how EDU structures group together: can we find patterns that recur across both languages, or will we observe distinct clusters unique to each language? This touches on a fundamental question: can we assume that EDU segmentation and syntax are consistent across languages, or are they inherently tied to each language's grammar?

Answering these questions requires combining insights from linguistic theory, computational methods, and empirical analysis. This work is grounded in the tradition of studying discourse organisation and the roles of units in structuring texts. It builds on the concepts of \emph{RST} and related frameworks. On the computational side, the study uses automatic sentence parsing based on \emph{Universal Dependencies}, enabling cross-linguistic comparison. 

The practical component of the project involves working with corpora of German and Russian texts that have been manually annotated for discourse structure. The German data is based on the work of Sara Shahmohammadi and Manfred Stede, who developed and described \emph{Discourse Parsing for German with new RST Corpora} (2024). The Russian data is drawn from corpora provided by HSE University, Skoltech University, RUDN University, and the Federal Research Centre ``Computer Science and Control'' of the Russian Academy of Sciences, developed by Ivan Smirnov, Svetlana Toldova, Dina Pisarevskaya, Maria Kobozeva, Artem Shelmanov, Elena Chistova, and Margarita Ananyev. By parsing these corpora computationally and analysing the resulting structures, the project seeks to uncover patterns and compare them across languages.

The methodology combines corpus linguistics, statistical analysis, and machine learning. EDUs are extracted from manually annotated RST corpora and processed through \texttt{spaCy} models to generate CoNLL-U formatted dependency parses. Syntactic features including EDU length, complexity, part-of-speech distributions, dependency relation frequencies, and structural patterns are then collected and compared across languages using statistical tests to detect differences and understand variation.

One notable contribution of this research is its groundwork for multilingual discourse processing, which has implications for machine translation, cross-lingual summarisation, and the development of multilingual parsers.

This thesis lies at the intersection of discourse theory, syntax, morphology, and computational linguistics. It seeks to determine whether EDU dependency structures follow universal principles or are shaped by the grammar of each language. Through extensive analysis, clustering, and comparison, this work aims to deepen our understanding of discourse organisation and to inform the development of more robust multilingual natural language processing tools. Access to the study is provided via the following link, and the material is freely available under the terms of the MIT License: \href{https://github.com/b-bblc/bachelor_thesis}{GitHub Repository}.

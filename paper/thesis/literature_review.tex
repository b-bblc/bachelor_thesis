\section{Literature Review}

The research foundation of this study relies on the fundamental sections, which are investigated in this chapter. To comprehend the study's results, it is necessary to understand topics that include Rhetorical Structure Theory (RST) for text structure analysis, Elementary Discourse Units (EDUs) as discourse building blocks, dependency parsing for sentence analysis, and clustering methods for linguistic pattern detection. The research aims to demonstrate various methods that reveal distinctive organisational structures of information systems within languages.

\subsection{Rhetorical Structure Theory (RST)}

Firstly, we will explore the meaning of Rhetorical Structure Theory, which serves as a method for studying the organisational patterns within written texts. The fundamental theory demonstrates that organised texts include elements which extend past single sentences because their structural components maintain relationships with each other. RST (Rhetorical Structure Theory) identifies approximately 25 distinct types of relationships, such as Cause, Contrast, and Elaboration, that serve to connect various segments of text. These relationships provide a framework for understanding the underlying structure and coherence of written discourse, thereby facilitating a deeper analysis of how information is interconnected within a textual context. 

However, the development of automated systems for analysing discourse structure remains challenging, although the outcomes of this work are highly valuable and contribute significantly to a deeper understanding of language processing. Pioneering work by Marcu (1997) established the foundational framework for computational rhetorical structure theory (RST) parsers, which continues to influence contemporary methodologies for evaluating discourse parsing systems. His contributions not only provided a systematic approach to discourse analysis but also delineated the criteria by which these systems could be assessed in terms of efficacy and accuracy. Moreover, in subsequent years, researchers such as Feng and Hirst (2014) advanced the field by demonstrating that incorporating syntactic information—particularly through the application of dependency parsing features—can markedly enhance parser performance. This evidence underscores the integral role of syntactic context in the disambiguation and interpretation of discourse structures, thereby facilitating more nuanced analytical frameworks. 

This lack of comparable resources has not only limited the development of RST parsers for many languages but also constrained researchers' ability to compare discourse structures across linguistic contexts systematically. Nevertheless, cross-linguistic studies that have been conducted reveal fascinating insights into how discourse relations manifest differently across languages. One of the most intriguing discoveries from cross-linguistic RST research is that, while various types of relationships exist across languages, their usage frequency and expression differ significantly. Taboada (2006) observed this phenomenon when comparing Spanish and English; similar trends are evident in this study. 

In the case of German, the Potsdam Commentary Corpus (Stede \& Neumann, 2014) indicated that German texts employ Elaboration and Contrast relationships differently than their English counterparts, likely due to varying writing conventions. Furthermore, the Russian RST-Treebank (Toldova et al., 2017) revealed distinct patterns in the way causal relationships are articulated in Russian, suggesting that grammatical differences between languages influence the organisation of discourse and will be discussed in more detail in the context of this study. 

\subsection{Elementary Discourse Units (EDUs)}

The construction of discourse structures depends on Elementary Discourse Units, which serve as their basic building blocks. To move forward, it is necessary to examine more closely how EDUs are understood and applied in discourse analysis. As mentioned above, Elementary Discourse Units are the smallest pieces of text that can participate in discourse relationships. The study by Carlson et al. (2003) suggests that EDUs are essentially ``clauses or clause-like units'' that can't be broken down into smaller, meaningful discourse pieces. 

The initial setup of EDU boundaries appears simple but becomes challenging to put into practice when implementing them in real-world situations. The main challenge is that EDU boundaries do not always match up perfectly with grammatical clause boundaries. A single clause in the text may consist of multiple discourse units, while different clauses can work together as one discourse unit. Task ambiguity leads to difficulties for human annotators who work on data labeling, as well as for computer systems that attempt to perform automated annotation. 

The manual EDU segmentation process requires trained annotators who follow established guidelines and procedures. The RST Discourse Treebank demonstrates an 85--90\% agreement level between annotators which proves high consistency but shows that human interpretation continues to affect the process. The automatic segmentation of EDU is more difficult. The best systems achieve about 75--85\% accuracy compared to human annotations. Soricut and Marcu (2003) established in their initial research that punctuation, along with discourse markers (such as "however" and "because") and syntactic boundaries, serve as a primary indicator for identifying EDU boundaries. 

Although early studies emphasised surface cues such as punctuation, discourse markers, and syntactic boundaries, subsequent research has shown that incorporating deeper syntactic information can substantially improve segmentation accuracy. The study by Fisher and Roark (2007) showed that dependency parsing features improve segmentation results most when analyzing complex sentences with multiple nested clauses. 

In cross-linguistic comparisons, languages organise their information content through EDUs in different ways. German EDUs tend to be longer and more syntactically complex than English ones, partly due to the greater flexibility of German word order and the more complicated nature of German noun phrases. Russian educational organisations have their own unique operational procedures. Pisarevskaya et al. (2017) identified specific features of Russian educational discourse units with regard to the placement of finite verbs and the use of participle constructions. 

Russian speakers can include more details in their individual EDUs because the language contains an extensive morphological system, which differs from languages with basic morphological structures, such as the Germanic languages. While such cross-linguistic differences highlight the variability in EDU length and structure, they also underscore the need for an analytical framework that can operate independently of word order and language-specific syntactic conventions. 

\subsection{Dependency Parsing}

This is where dependency grammar becomes particularly valuable. The dependency grammar examines word-to-word connections instead of using hierarchical structures to construct phrase structures. The dependency analysis method converts each sentence into a tree structure, which shows words through labelled arrows that show their grammatical relationships (such as subject, object, and modifier). The main advantage of dependency grammar for research in dependency structures in Elementary Discourse Units is that it works well across different languages. The fundamental grammatical connections between subject and object exist similarly in all languages, although their word order and phrase structure differ significantly. 

Modern dependency parsers work pretty well. Standard benchmark tests show that the top systems for English and German language processing achieve more than 95\% accuracy. The model demonstrates high accuracy which enables its deployment for dependable use in my research and subsequent applications. However, the accuracy of parsing varies when processing different languages. Research studies from recent times show that parsing errors remain stable when analysing equivalent contexts, which improves the reliability of studies that compare different things. 

Using dependency parsing to study discourse phenomena is a relatively new approach. Most work has focused on identifying discourse connectives and how they connect their arguments. The research by Pitler and Nenkova (2009) showed that dependency tree features improve the detection of discourse relationships. More relevant to dependency structure analysis is the study from Kong et al. (2014), who demonstrated that dependency parsing-based complexity metrics, which measure tree depth, align with human assessments of text coherence. The results indicate that syntactic complexity features identify essential organisational elements in text structure. 

\subsection{Clustering Methods in Linguistics}

The study of syntactic patterns in EDUs through dependency features has not received sufficient attention in previous research because researchers have concentrated on discourse unit connections, which is why this approach of using dependency features to analyse patterns within EDUs represents a novel application. Scientists use clustering data analysis to find hidden data patterns that do not need predefined search criteria. The method proves helpful in linguistics because it helps researchers identify patterns which traditional categories fail to detect and enables them to find new patterns that enhance current theoretical frameworks. 

The theoretical basis for clustering in linguistics stems from usage-based approaches to language, which were first introduced by Tomasello (2003). According to this theory, linguistic categories develop through patterns of usage instead of following innate rules. The clustering method enables researchers to identify statistical patterns which native speakers automatically learn and use in their spoken language. 

Clustering has been successfully applied to various syntactic problems. Parisien and Stevenson (2010) used clustering to study verb classes based on their syntactic patterns, which produced categories that combined elements from traditional semantic groups but also presented distinct new differences. While such results highlight the potential of clustering within a single linguistic system, extending this approach to cross-linguistic contexts introduces additional challenges. 

The application of clustering methods to different languages requires addressing particular difficulties. Different languages display systematic variations in feature frequencies, which stem from their typological characteristics instead of actual discourse differences. The process requires both feature normalization and validation procedures that need to be implemented with precision. Research on cross-linguistic clustering has shown promising results, although it faces various obstacles. Feng and Hirst (2012) analyzed discourse relation distribution between languages by using clustering methods, which demonstrated that languages structure their discourse to form particular clusters. 

The primary challenge in clustering research is distinguishing genuine patterns from statistical noise. Evaluation combines statistical validation (silhouette scores) with expert assessment of detected patterns. Given the cross-linguistic comparison, validation is crucial to ensure that identified patterns reflect actual discourse organization differences rather than artifacts of different corpora or parsing systems.

\subsection{Cross-Linguistic Perspectives and Research Gaps}

This study focuses on German (a V2 language with rigid word order) and Russian (a morphologically rich language with flexible syntax). Such structural contrasts influence how information is distributed across EDUs. Cross-linguistic discourse parsing faces challenges related to data comparability and parser quality. Two major gaps in the literature are identified: the underexplored connection between EDU-internal syntax and higher-level discourse structure, and the lack of multilingual validation of discourse models. This work aims to address both by analyzing dependency structures in German and Russian EDUs.